% Generated by Sphinx.
\def\sphinxdocclass{report}
\documentclass[letterpaper,10pt,english]{sphinxmanual}
\usepackage[utf8]{inputenc}
\DeclareUnicodeCharacter{00A0}{\nobreakspace}
\usepackage{cmap}
\usepackage[T1]{fontenc}
\usepackage{babel}
\usepackage{times}
\usepackage[Bjarne]{fncychap}
\usepackage{longtable}
\usepackage{sphinx}
\usepackage{multirow}

\addto\captionsenglish{\renewcommand{\figurename}{Fig. }}
\addto\captionsenglish{\renewcommand{\tablename}{Table }}
\floatname{literal-block}{Listing }



\title{myDocApp Documentation}
\date{January 16, 2016}
\release{1.0}
\author{luminita}
\newcommand{\sphinxlogo}{}
\renewcommand{\releasename}{Release}
\makeindex

\makeatletter
\def\PYG@reset{\let\PYG@it=\relax \let\PYG@bf=\relax%
    \let\PYG@ul=\relax \let\PYG@tc=\relax%
    \let\PYG@bc=\relax \let\PYG@ff=\relax}
\def\PYG@tok#1{\csname PYG@tok@#1\endcsname}
\def\PYG@toks#1+{\ifx\relax#1\empty\else%
    \PYG@tok{#1}\expandafter\PYG@toks\fi}
\def\PYG@do#1{\PYG@bc{\PYG@tc{\PYG@ul{%
    \PYG@it{\PYG@bf{\PYG@ff{#1}}}}}}}
\def\PYG#1#2{\PYG@reset\PYG@toks#1+\relax+\PYG@do{#2}}

\expandafter\def\csname PYG@tok@c1\endcsname{\let\PYG@it=\textit\def\PYG@tc##1{\textcolor[rgb]{0.25,0.50,0.56}{##1}}}
\expandafter\def\csname PYG@tok@gh\endcsname{\let\PYG@bf=\textbf\def\PYG@tc##1{\textcolor[rgb]{0.00,0.00,0.50}{##1}}}
\expandafter\def\csname PYG@tok@mh\endcsname{\def\PYG@tc##1{\textcolor[rgb]{0.13,0.50,0.31}{##1}}}
\expandafter\def\csname PYG@tok@gi\endcsname{\def\PYG@tc##1{\textcolor[rgb]{0.00,0.63,0.00}{##1}}}
\expandafter\def\csname PYG@tok@mi\endcsname{\def\PYG@tc##1{\textcolor[rgb]{0.13,0.50,0.31}{##1}}}
\expandafter\def\csname PYG@tok@kr\endcsname{\let\PYG@bf=\textbf\def\PYG@tc##1{\textcolor[rgb]{0.00,0.44,0.13}{##1}}}
\expandafter\def\csname PYG@tok@ne\endcsname{\def\PYG@tc##1{\textcolor[rgb]{0.00,0.44,0.13}{##1}}}
\expandafter\def\csname PYG@tok@sh\endcsname{\def\PYG@tc##1{\textcolor[rgb]{0.25,0.44,0.63}{##1}}}
\expandafter\def\csname PYG@tok@gs\endcsname{\let\PYG@bf=\textbf}
\expandafter\def\csname PYG@tok@bp\endcsname{\def\PYG@tc##1{\textcolor[rgb]{0.00,0.44,0.13}{##1}}}
\expandafter\def\csname PYG@tok@il\endcsname{\def\PYG@tc##1{\textcolor[rgb]{0.13,0.50,0.31}{##1}}}
\expandafter\def\csname PYG@tok@k\endcsname{\let\PYG@bf=\textbf\def\PYG@tc##1{\textcolor[rgb]{0.00,0.44,0.13}{##1}}}
\expandafter\def\csname PYG@tok@go\endcsname{\def\PYG@tc##1{\textcolor[rgb]{0.20,0.20,0.20}{##1}}}
\expandafter\def\csname PYG@tok@ge\endcsname{\let\PYG@it=\textit}
\expandafter\def\csname PYG@tok@sx\endcsname{\def\PYG@tc##1{\textcolor[rgb]{0.78,0.36,0.04}{##1}}}
\expandafter\def\csname PYG@tok@sc\endcsname{\def\PYG@tc##1{\textcolor[rgb]{0.25,0.44,0.63}{##1}}}
\expandafter\def\csname PYG@tok@vg\endcsname{\def\PYG@tc##1{\textcolor[rgb]{0.73,0.38,0.84}{##1}}}
\expandafter\def\csname PYG@tok@sb\endcsname{\def\PYG@tc##1{\textcolor[rgb]{0.25,0.44,0.63}{##1}}}
\expandafter\def\csname PYG@tok@s\endcsname{\def\PYG@tc##1{\textcolor[rgb]{0.25,0.44,0.63}{##1}}}
\expandafter\def\csname PYG@tok@sd\endcsname{\let\PYG@it=\textit\def\PYG@tc##1{\textcolor[rgb]{0.25,0.44,0.63}{##1}}}
\expandafter\def\csname PYG@tok@nl\endcsname{\let\PYG@bf=\textbf\def\PYG@tc##1{\textcolor[rgb]{0.00,0.13,0.44}{##1}}}
\expandafter\def\csname PYG@tok@nf\endcsname{\def\PYG@tc##1{\textcolor[rgb]{0.02,0.16,0.49}{##1}}}
\expandafter\def\csname PYG@tok@nc\endcsname{\let\PYG@bf=\textbf\def\PYG@tc##1{\textcolor[rgb]{0.05,0.52,0.71}{##1}}}
\expandafter\def\csname PYG@tok@s1\endcsname{\def\PYG@tc##1{\textcolor[rgb]{0.25,0.44,0.63}{##1}}}
\expandafter\def\csname PYG@tok@mf\endcsname{\def\PYG@tc##1{\textcolor[rgb]{0.13,0.50,0.31}{##1}}}
\expandafter\def\csname PYG@tok@sr\endcsname{\def\PYG@tc##1{\textcolor[rgb]{0.14,0.33,0.53}{##1}}}
\expandafter\def\csname PYG@tok@gp\endcsname{\let\PYG@bf=\textbf\def\PYG@tc##1{\textcolor[rgb]{0.78,0.36,0.04}{##1}}}
\expandafter\def\csname PYG@tok@nt\endcsname{\let\PYG@bf=\textbf\def\PYG@tc##1{\textcolor[rgb]{0.02,0.16,0.45}{##1}}}
\expandafter\def\csname PYG@tok@no\endcsname{\def\PYG@tc##1{\textcolor[rgb]{0.38,0.68,0.84}{##1}}}
\expandafter\def\csname PYG@tok@vi\endcsname{\def\PYG@tc##1{\textcolor[rgb]{0.73,0.38,0.84}{##1}}}
\expandafter\def\csname PYG@tok@m\endcsname{\def\PYG@tc##1{\textcolor[rgb]{0.13,0.50,0.31}{##1}}}
\expandafter\def\csname PYG@tok@kn\endcsname{\let\PYG@bf=\textbf\def\PYG@tc##1{\textcolor[rgb]{0.00,0.44,0.13}{##1}}}
\expandafter\def\csname PYG@tok@kt\endcsname{\def\PYG@tc##1{\textcolor[rgb]{0.56,0.13,0.00}{##1}}}
\expandafter\def\csname PYG@tok@gt\endcsname{\def\PYG@tc##1{\textcolor[rgb]{0.00,0.27,0.87}{##1}}}
\expandafter\def\csname PYG@tok@nd\endcsname{\let\PYG@bf=\textbf\def\PYG@tc##1{\textcolor[rgb]{0.33,0.33,0.33}{##1}}}
\expandafter\def\csname PYG@tok@cm\endcsname{\let\PYG@it=\textit\def\PYG@tc##1{\textcolor[rgb]{0.25,0.50,0.56}{##1}}}
\expandafter\def\csname PYG@tok@mo\endcsname{\def\PYG@tc##1{\textcolor[rgb]{0.13,0.50,0.31}{##1}}}
\expandafter\def\csname PYG@tok@err\endcsname{\def\PYG@bc##1{\setlength{\fboxsep}{0pt}\fcolorbox[rgb]{1.00,0.00,0.00}{1,1,1}{\strut ##1}}}
\expandafter\def\csname PYG@tok@nv\endcsname{\def\PYG@tc##1{\textcolor[rgb]{0.73,0.38,0.84}{##1}}}
\expandafter\def\csname PYG@tok@kp\endcsname{\def\PYG@tc##1{\textcolor[rgb]{0.00,0.44,0.13}{##1}}}
\expandafter\def\csname PYG@tok@gu\endcsname{\let\PYG@bf=\textbf\def\PYG@tc##1{\textcolor[rgb]{0.50,0.00,0.50}{##1}}}
\expandafter\def\csname PYG@tok@gd\endcsname{\def\PYG@tc##1{\textcolor[rgb]{0.63,0.00,0.00}{##1}}}
\expandafter\def\csname PYG@tok@c\endcsname{\let\PYG@it=\textit\def\PYG@tc##1{\textcolor[rgb]{0.25,0.50,0.56}{##1}}}
\expandafter\def\csname PYG@tok@cs\endcsname{\def\PYG@tc##1{\textcolor[rgb]{0.25,0.50,0.56}{##1}}\def\PYG@bc##1{\setlength{\fboxsep}{0pt}\colorbox[rgb]{1.00,0.94,0.94}{\strut ##1}}}
\expandafter\def\csname PYG@tok@cp\endcsname{\def\PYG@tc##1{\textcolor[rgb]{0.00,0.44,0.13}{##1}}}
\expandafter\def\csname PYG@tok@mb\endcsname{\def\PYG@tc##1{\textcolor[rgb]{0.13,0.50,0.31}{##1}}}
\expandafter\def\csname PYG@tok@nb\endcsname{\def\PYG@tc##1{\textcolor[rgb]{0.00,0.44,0.13}{##1}}}
\expandafter\def\csname PYG@tok@kd\endcsname{\let\PYG@bf=\textbf\def\PYG@tc##1{\textcolor[rgb]{0.00,0.44,0.13}{##1}}}
\expandafter\def\csname PYG@tok@s2\endcsname{\def\PYG@tc##1{\textcolor[rgb]{0.25,0.44,0.63}{##1}}}
\expandafter\def\csname PYG@tok@w\endcsname{\def\PYG@tc##1{\textcolor[rgb]{0.73,0.73,0.73}{##1}}}
\expandafter\def\csname PYG@tok@o\endcsname{\def\PYG@tc##1{\textcolor[rgb]{0.40,0.40,0.40}{##1}}}
\expandafter\def\csname PYG@tok@si\endcsname{\let\PYG@it=\textit\def\PYG@tc##1{\textcolor[rgb]{0.44,0.63,0.82}{##1}}}
\expandafter\def\csname PYG@tok@na\endcsname{\def\PYG@tc##1{\textcolor[rgb]{0.25,0.44,0.63}{##1}}}
\expandafter\def\csname PYG@tok@nn\endcsname{\let\PYG@bf=\textbf\def\PYG@tc##1{\textcolor[rgb]{0.05,0.52,0.71}{##1}}}
\expandafter\def\csname PYG@tok@se\endcsname{\let\PYG@bf=\textbf\def\PYG@tc##1{\textcolor[rgb]{0.25,0.44,0.63}{##1}}}
\expandafter\def\csname PYG@tok@ni\endcsname{\let\PYG@bf=\textbf\def\PYG@tc##1{\textcolor[rgb]{0.84,0.33,0.22}{##1}}}
\expandafter\def\csname PYG@tok@vc\endcsname{\def\PYG@tc##1{\textcolor[rgb]{0.73,0.38,0.84}{##1}}}
\expandafter\def\csname PYG@tok@ss\endcsname{\def\PYG@tc##1{\textcolor[rgb]{0.32,0.47,0.09}{##1}}}
\expandafter\def\csname PYG@tok@kc\endcsname{\let\PYG@bf=\textbf\def\PYG@tc##1{\textcolor[rgb]{0.00,0.44,0.13}{##1}}}
\expandafter\def\csname PYG@tok@ow\endcsname{\let\PYG@bf=\textbf\def\PYG@tc##1{\textcolor[rgb]{0.00,0.44,0.13}{##1}}}
\expandafter\def\csname PYG@tok@gr\endcsname{\def\PYG@tc##1{\textcolor[rgb]{1.00,0.00,0.00}{##1}}}

\def\PYGZbs{\char`\\}
\def\PYGZus{\char`\_}
\def\PYGZob{\char`\{}
\def\PYGZcb{\char`\}}
\def\PYGZca{\char`\^}
\def\PYGZam{\char`\&}
\def\PYGZlt{\char`\<}
\def\PYGZgt{\char`\>}
\def\PYGZsh{\char`\#}
\def\PYGZpc{\char`\%}
\def\PYGZdl{\char`\$}
\def\PYGZhy{\char`\-}
\def\PYGZsq{\char`\'}
\def\PYGZdq{\char`\"}
\def\PYGZti{\char`\~}
% for compatibility with earlier versions
\def\PYGZat{@}
\def\PYGZlb{[}
\def\PYGZrb{]}
\makeatother

\renewcommand\PYGZsq{\textquotesingle}

\begin{document}

\maketitle
\tableofcontents
\phantomsection\label{index::doc}


Contents:


\chapter{DAO Package}
\label{dao::doc}\label{dao:welcome-to-mydocapp-s-documentation}\label{dao:dao-package}
The DAO class provides an abstract interface to the database and some specific data operations without exposing details of the database.
\phantomsection\label{dao:module-doctors.dao}\index{doctors.dao (module)}\index{AppointmentDAO (class in doctors.dao)}

\begin{fulllineitems}
\phantomsection\label{dao:doctors.dao.AppointmentDAO}\pysigline{\strong{class }\code{doctors.dao.}\bfcode{AppointmentDAO}}
abstract interface to database object: Specialization
\index{create() (doctors.dao.AppointmentDAO method)}

\begin{fulllineitems}
\phantomsection\label{dao:doctors.dao.AppointmentDAO.create}\pysiglinewithargsret{\bfcode{create}}{\emph{patient}, \emph{doctor}, \emph{date}, \emph{valid}}{}
method which creates new appointments specifing the Patient which created, Doctor which was booked, date of creation and validation status

\end{fulllineitems}

\index{getByDoctor() (doctors.dao.AppointmentDAO method)}

\begin{fulllineitems}
\phantomsection\label{dao:doctors.dao.AppointmentDAO.getByDoctor}\pysiglinewithargsret{\bfcode{getByDoctor}}{\emph{doctor}}{}
method which returns from database the Appointments made by the specified Doctor object

\end{fulllineitems}

\index{getByID() (doctors.dao.AppointmentDAO method)}

\begin{fulllineitems}
\phantomsection\label{dao:doctors.dao.AppointmentDAO.getByID}\pysiglinewithargsret{\bfcode{getByID}}{\emph{ID}}{}
method which returns from database the Appointments made by the specified appointment's ID (primary key)

\end{fulllineitems}

\index{getByPatient() (doctors.dao.AppointmentDAO method)}

\begin{fulllineitems}
\phantomsection\label{dao:doctors.dao.AppointmentDAO.getByPatient}\pysiglinewithargsret{\bfcode{getByPatient}}{\emph{patient}}{}
method which returns from database the Appointments made by the specified Patient object

\end{fulllineitems}

\index{getTakenDays() (doctors.dao.AppointmentDAO method)}

\begin{fulllineitems}
\phantomsection\label{dao:doctors.dao.AppointmentDAO.getTakenDays}\pysiglinewithargsret{\bfcode{getTakenDays}}{\emph{doctor}, \emph{date}}{}
method which returns from database the days of the specified Doctor timetable which are allready taken for appointments

\end{fulllineitems}


\end{fulllineitems}

\index{AuthentificationDAO (class in doctors.dao)}

\begin{fulllineitems}
\phantomsection\label{dao:doctors.dao.AuthentificationDAO}\pysigline{\strong{class }\code{doctors.dao.}\bfcode{AuthentificationDAO}}
class working with logging in the myDoc System
\index{findUserByUsername() (doctors.dao.AuthentificationDAO method)}

\begin{fulllineitems}
\phantomsection\label{dao:doctors.dao.AuthentificationDAO.findUserByUsername}\pysiglinewithargsret{\bfcode{findUserByUsername}}{\emph{user}, \emph{password}}{}
method for cheking if the credentials entered by user are in the database (i.e. if such a user exists).

\end{fulllineitems}


\end{fulllineitems}

\index{BaseDAO (class in doctors.dao)}

\begin{fulllineitems}
\phantomsection\label{dao:doctors.dao.BaseDAO}\pysigline{\strong{class }\code{doctors.dao.}\bfcode{BaseDAO}}
BaseDAO contains all the abstracts methods that should be implemented in the Controller.
\index{delete() (doctors.dao.BaseDAO method)}

\begin{fulllineitems}
\phantomsection\label{dao:doctors.dao.BaseDAO.delete}\pysiglinewithargsret{\bfcode{delete}}{}{}
abstract method for deleting objects from database

\end{fulllineitems}

\index{getAll() (doctors.dao.BaseDAO method)}

\begin{fulllineitems}
\phantomsection\label{dao:doctors.dao.BaseDAO.getAll}\pysiglinewithargsret{\bfcode{getAll}}{}{}
abstract method which returns all specified types of objects from database

\end{fulllineitems}

\index{getById() (doctors.dao.BaseDAO method)}

\begin{fulllineitems}
\phantomsection\label{dao:doctors.dao.BaseDAO.getById}\pysiglinewithargsret{\bfcode{getById}}{\emph{ID}}{}
abstract method which returns the objects from database which has the specified ID (primary key)

\end{fulllineitems}

\index{save() (doctors.dao.BaseDAO method)}

\begin{fulllineitems}
\phantomsection\label{dao:doctors.dao.BaseDAO.save}\pysiglinewithargsret{\bfcode{save}}{}{}
abstractmethod methods which save created object to database

\end{fulllineitems}


\end{fulllineitems}

\index{DoctorDAO (class in doctors.dao)}

\begin{fulllineitems}
\phantomsection\label{dao:doctors.dao.DoctorDAO}\pysigline{\strong{class }\code{doctors.dao.}\bfcode{DoctorDAO}}
abstract interface to database object: Doctor
\index{create() (doctors.dao.DoctorDAO method)}

\begin{fulllineitems}
\phantomsection\label{dao:doctors.dao.DoctorDAO.create}\pysiglinewithargsret{\bfcode{create}}{\emph{name}, \emph{spec}}{}
creates new objects of type Patient by and specialization

\end{fulllineitems}

\index{getByCriteria() (doctors.dao.DoctorDAO method)}

\begin{fulllineitems}
\phantomsection\label{dao:doctors.dao.DoctorDAO.getByCriteria}\pysiglinewithargsret{\bfcode{getByCriteria}}{\emph{spec}, \emph{zipcode}}{}
method which returns from database Doctor objects with specified specialization and zip\_code

\end{fulllineitems}

\index{getByFilter() (doctors.dao.DoctorDAO method)}

\begin{fulllineitems}
\phantomsection\label{dao:doctors.dao.DoctorDAO.getByFilter}\pysiglinewithargsret{\bfcode{getByFilter}}{\emph{gen}}{}
method which returns from database Doctor objects with specified gender

\end{fulllineitems}


\end{fulllineitems}

\index{HospitalDAO (class in doctors.dao)}

\begin{fulllineitems}
\phantomsection\label{dao:doctors.dao.HospitalDAO}\pysigline{\strong{class }\code{doctors.dao.}\bfcode{HospitalDAO}}
abstract interface to database object: Specialization
\index{getByDoctor() (doctors.dao.HospitalDAO method)}

\begin{fulllineitems}
\phantomsection\label{dao:doctors.dao.HospitalDAO.getByDoctor}\pysiglinewithargsret{\bfcode{getByDoctor}}{\emph{doctor}}{}
method which returns from database the Hospital of specified Doctor object

\end{fulllineitems}


\end{fulllineitems}

\index{PatientDAO (class in doctors.dao)}

\begin{fulllineitems}
\phantomsection\label{dao:doctors.dao.PatientDAO}\pysigline{\strong{class }\code{doctors.dao.}\bfcode{PatientDAO}}
abstract interface to database object: Patient
\index{create() (doctors.dao.PatientDAO method)}

\begin{fulllineitems}
\phantomsection\label{dao:doctors.dao.PatientDAO.create}\pysiglinewithargsret{\bfcode{create}}{\emph{name}}{}
creates new objects of type Patient by name

\end{fulllineitems}


\end{fulllineitems}

\index{SpecializationDAO (class in doctors.dao)}

\begin{fulllineitems}
\phantomsection\label{dao:doctors.dao.SpecializationDAO}\pysigline{\strong{class }\code{doctors.dao.}\bfcode{SpecializationDAO}}
abstract interface to database object: Specialization
\index{getByDoctor() (doctors.dao.SpecializationDAO method)}

\begin{fulllineitems}
\phantomsection\label{dao:doctors.dao.SpecializationDAO.getByDoctor}\pysiglinewithargsret{\bfcode{getByDoctor}}{\emph{doctor}}{}
method which returns from database type of Specialization of specified Doctor object

\end{fulllineitems}


\end{fulllineitems}



\chapter{Controllers Package}
\label{controllers:controllers-package}\label{controllers::doc}
The controllers specify functionalities which are provided from the Model layer for the View layer.
\phantomsection\label{controllers:module-doctors.controllers}\index{doctors.controllers (module)}\index{AppointmentController (class in doctors.controllers)}

\begin{fulllineitems}
\phantomsection\label{controllers:doctors.controllers.AppointmentController}\pysigline{\strong{class }\code{doctors.controllers.}\bfcode{AppointmentController}}
AppointmentController class deals with implementing the methods specified in AppointmentDAO class

\end{fulllineitems}

\index{DoctorController (class in doctors.controllers)}

\begin{fulllineitems}
\phantomsection\label{controllers:doctors.controllers.DoctorController}\pysigline{\strong{class }\code{doctors.controllers.}\bfcode{DoctorController}}
DoctorController class deals with implementing the methods specified in DoctorDAO class
\index{getById() (doctors.controllers.DoctorController method)}

\begin{fulllineitems}
\phantomsection\label{controllers:doctors.controllers.DoctorController.getById}\pysiglinewithargsret{\bfcode{getById}}{\emph{doc\_id}}{}
method that retrieves  Doctor objects from database by specified ID

\end{fulllineitems}

\index{getBySpecHospId() (doctors.controllers.DoctorController method)}

\begin{fulllineitems}
\phantomsection\label{controllers:doctors.controllers.DoctorController.getBySpecHospId}\pysiglinewithargsret{\bfcode{getBySpecHospId}}{\emph{spec\_id}, \emph{hosp\_id}}{}
method that retrieves  Doctor objects from database by specified specialization ID and Hospital ID

\end{fulllineitems}


\end{fulllineitems}

\index{HospitalController (class in doctors.controllers)}

\begin{fulllineitems}
\phantomsection\label{controllers:doctors.controllers.HospitalController}\pysigline{\strong{class }\code{doctors.controllers.}\bfcode{HospitalController}}
HospitalController class deals with implementing the methods specified in HospitalDAO class
\index{getAll() (doctors.controllers.HospitalController method)}

\begin{fulllineitems}
\phantomsection\label{controllers:doctors.controllers.HospitalController.getAll}\pysiglinewithargsret{\bfcode{getAll}}{}{}
method that retrieves all Hospital objects from database

\end{fulllineitems}

\index{getById() (doctors.controllers.HospitalController method)}

\begin{fulllineitems}
\phantomsection\label{controllers:doctors.controllers.HospitalController.getById}\pysiglinewithargsret{\bfcode{getById}}{\emph{ID}}{}
method that retrieves Hospital objects from database by specified ID

\end{fulllineitems}


\end{fulllineitems}

\index{PatientController (class in doctors.controllers)}

\begin{fulllineitems}
\phantomsection\label{controllers:doctors.controllers.PatientController}\pysigline{\strong{class }\code{doctors.controllers.}\bfcode{PatientController}}
PatientController class deals with implementing the methods specified in PatientDAO class

\end{fulllineitems}

\index{SpecializationController (class in doctors.controllers)}

\begin{fulllineitems}
\phantomsection\label{controllers:doctors.controllers.SpecializationController}\pysigline{\strong{class }\code{doctors.controllers.}\bfcode{SpecializationController}}
SpecializationController class deals with implementing the methods specified in SpecializationDAO class
\index{getAll() (doctors.controllers.SpecializationController method)}

\begin{fulllineitems}
\phantomsection\label{controllers:doctors.controllers.SpecializationController.getAll}\pysiglinewithargsret{\bfcode{getAll}}{}{}
method that retrieves all specializations types from database

\end{fulllineitems}

\index{getById() (doctors.controllers.SpecializationController method)}

\begin{fulllineitems}
\phantomsection\label{controllers:doctors.controllers.SpecializationController.getById}\pysiglinewithargsret{\bfcode{getById}}{\emph{ID}}{}
method that retrieves specializations types from database by specified ID

\end{fulllineitems}


\end{fulllineitems}



\chapter{Managers Package}
\label{managers:module-doctors.managers}\label{managers:managers-package}\label{managers::doc}\index{doctors.managers (module)}\index{DoctorManager (class in doctors.managers)}

\begin{fulllineitems}
\phantomsection\label{managers:doctors.managers.DoctorManager}\pysigline{\strong{class }\code{doctors.managers.}\bfcode{DoctorManager}}
DoctorManager class
\index{create\_patient() (doctors.managers.DoctorManager method)}

\begin{fulllineitems}
\phantomsection\label{managers:doctors.managers.DoctorManager.create_patient}\pysiglinewithargsret{\bfcode{create\_patient}}{\emph{name}, \emph{password=None}}{}
method for careting new users with the role Doctor in the System

\end{fulllineitems}


\end{fulllineitems}

\index{PatientManager (class in doctors.managers)}

\begin{fulllineitems}
\phantomsection\label{managers:doctors.managers.PatientManager}\pysigline{\strong{class }\code{doctors.managers.}\bfcode{PatientManager}}
PatientManager class
\index{create\_patient() (doctors.managers.PatientManager method)}

\begin{fulllineitems}
\phantomsection\label{managers:doctors.managers.PatientManager.create_patient}\pysiglinewithargsret{\bfcode{create\_patient}}{\emph{name}, \emph{password=None}}{}
method for careting new users with the role Patient in the System

\end{fulllineitems}


\end{fulllineitems}



\chapter{Forms Package}
\label{forms:forms-package}\label{forms::doc}
``Forms describe a form and determines how it works and appears. A form class’s fields map to HTML form \textless{}input\textgreater{} elements (/myDoc/doctors/templates.''
\phantomsection\label{forms:module-doctors.forms}\index{doctors.forms (module)}\index{BookDoc (class in doctors.forms)}

\begin{fulllineitems}
\phantomsection\label{forms:doctors.forms.BookDoc}\pysiglinewithargsret{\strong{class }\code{doctors.forms.}\bfcode{BookDoc}}{\emph{data=None}, \emph{files=None}, \emph{auto\_id='id\_\%s'}, \emph{prefix=None}, \emph{initial=None}, \emph{error\_class=\textless{}class `django.forms.utils.ErrorList'\textgreater{}}, \emph{label\_suffix=None}, \emph{empty\_permitted=False}, \emph{field\_order=None}}{}
SearchDoc Class represents a form which allows user to fill it with required details about reservation. The form uses POST method and save the introduced data as a new Appointment Object in database

\end{fulllineitems}

\index{SearchDoc (class in doctors.forms)}

\begin{fulllineitems}
\phantomsection\label{forms:doctors.forms.SearchDoc}\pysiglinewithargsret{\strong{class }\code{doctors.forms.}\bfcode{SearchDoc}}{\emph{data=None}, \emph{files=None}, \emph{auto\_id='id\_\%s'}, \emph{prefix=None}, \emph{initial=None}, \emph{error\_class=\textless{}class `django.forms.utils.ErrorList'\textgreater{}}, \emph{label\_suffix=None}, \emph{empty\_permitted=False}, \emph{field\_order=None}}{}
SearchDoc Class represents a form which allows user to fill it with desired doctor's specialization and zip\_code. The form uses GET method and returns a list of Doctor objects from database results that fulfill the required search criteria

\end{fulllineitems}



\chapter{Models Package}
\label{models:models-package}\label{models::doc}
``A model is the single, definitive source of information about the data. It contains the essential fields and behaviors of the data that is storing in the myDoc System. Generally, each model maps to a single database table. The unique ID (primary key) are craeted aoutomatically.''
\phantomsection\label{models:module-doctors.models}\index{doctors.models (module)}\index{Appointment (class in doctors.models)}

\begin{fulllineitems}
\phantomsection\label{models:doctors.models.Appointment}\pysiglinewithargsret{\strong{class }\code{doctors.models.}\bfcode{Appointment}}{\emph{*args}, \emph{**kwargs}}{}
Stores the data about each Appointment. Each Appointment has the filds: patient which created the appointment, doctor at whom, date of creation and validation status

\end{fulllineitems}

\index{Doctor (class in doctors.models)}

\begin{fulllineitems}
\phantomsection\label{models:doctors.models.Doctor}\pysiglinewithargsret{\strong{class }\code{doctors.models.}\bfcode{Doctor}}{\emph{*args}, \emph{**kwargs}}{}
Stores the data about each Doctor. Each doctor is defined by name, rating, speciality, zip\_code, photo, gender and education. The field password is automatically added from the DoctorManager class

\end{fulllineitems}

\index{Hospital (class in doctors.models)}

\begin{fulllineitems}
\phantomsection\label{models:doctors.models.Hospital}\pysiglinewithargsret{\strong{class }\code{doctors.models.}\bfcode{Hospital}}{\emph{*args}, \emph{**kwargs}}{}
Stores data about each Hospital such that every one of them has an addres, zip\_code and name

\end{fulllineitems}

\index{Patient (class in doctors.models)}

\begin{fulllineitems}
\phantomsection\label{models:doctors.models.Patient}\pysiglinewithargsret{\strong{class }\code{doctors.models.}\bfcode{Patient}}{\emph{*args}, \emph{**kwargs}}{}
Stores the data about each Patient. Each doctor is defined by name, surname, insurance number, date of birth, email. The field password is automatically added from the DoctorManager class

\end{fulllineitems}

\index{Specialization (class in doctors.models)}

\begin{fulllineitems}
\phantomsection\label{models:doctors.models.Specialization}\pysiglinewithargsret{\strong{class }\code{doctors.models.}\bfcode{Specialization}}{\emph{*args}, \emph{**kwargs}}{}
Specialization model stores the data about the types of specialization

\end{fulllineitems}



\chapter{Views Package}
\label{views::doc}\label{views:views-package}
A view function, or view for short, is simply a Python function that takes a Web request and returns a Web response.
\phantomsection\label{views:module-doctors.views}\index{doctors.views (module)}\index{booking() (in module doctors.views)}

\begin{fulllineitems}
\phantomsection\label{views:doctors.views.booking}\pysiglinewithargsret{\code{doctors.views.}\bfcode{booking}}{\emph{request}, \emph{doctor\_id}}{}
Display the confirmation message about the successfull booking process

\end{fulllineitems}

\index{booking\_form() (in module doctors.views)}

\begin{fulllineitems}
\phantomsection\label{views:doctors.views.booking_form}\pysiglinewithargsret{\code{doctors.views.}\bfcode{booking\_form}}{\emph{request}, \emph{doctor\_id}}{}
Render the booking form. The form is rendering after the user requested to book a specific doctor.

\end{fulllineitems}

\index{home() (in module doctors.views)}

\begin{fulllineitems}
\phantomsection\label{views:doctors.views.home}\pysiglinewithargsret{\code{doctors.views.}\bfcode{home}}{\emph{request}}{}
Render the home page of the myDoc System

\end{fulllineitems}

\index{search() (in module doctors.views)}

\begin{fulllineitems}
\phantomsection\label{views:doctors.views.search}\pysiglinewithargsret{\code{doctors.views.}\bfcode{search}}{\emph{request}}{}
Render the page with the doctorlist based on the desired search requirements requested by user

\end{fulllineitems}

\index{showDoctorDetails() (in module doctors.views)}

\begin{fulllineitems}
\phantomsection\label{views:doctors.views.showDoctorDetails}\pysiglinewithargsret{\code{doctors.views.}\bfcode{showDoctorDetails}}{\emph{request}, \emph{doctor\_id}}{}
Render the page with the details about the doctor choosen by the user from the list of doctors

\end{fulllineitems}


\renewcommand{\indexname}{Index}
\printindex
\end{document}
